%% Template para dissertação/tese na classe UFBAthesis
%% versão 0.9.2
%% (c) 2005 Paulo G. S. Fonseca
%% (c) 2012 Antonio Terceiro
%% www.dcc.ufba.br/~terceiro/ufbathesis

\documentclass[phd]{ufbathesis}
\usepackage[utf8]{inputenc}

%% Preâmbulo:
%% coloque aqui o seu preâmbulo LaTeX, i.e., declaração de pacotes,
%% (re)definições de macros, medidas, etc.

\title{<TÍTULO DA OBRA>}
\date{<DATA DA DEFESA>}
\author{<NOME DO AUTOR>}
\adviser{<NOME DO(DA) ORIENTADOR(A)>}
\coadviser{<NOME DO(DA) CO-ORIENTADOR(A)>}

\begin{document}

%%
%% Parte pré-textual
%%
\frontmatter

% Folha de rosto
% Comente para ocultar
\frontpage

% Portada (apresentação)
% Comente para ocultar
\presentationpage

% Agradecimentos
% Se preferir, crie um arquivo à parte e o inclua via \include{}
\acknowledgements
<DIGITE OS AGRADECIMENTOS AQUI>

% Resumo em Português
% Se preferir, crie um arquivo à parte e o inclua via \include{}
\resumo
<DIGITE O RESUMO AQUI>
% Palavras-chave do resumo em Português
\begin{keywords}
<DIGITE AS PALAVRAS-CHAVE AQUI>
\end{keywords}

% Resumo em Inglês
% Se preferir, crie um arquivo à parte e o inclua via \include{}
\abstract
% Palavras-chave do resumo em Inglês
\begin{keywords}
<DIGITE AS PALAVRAS-CHAVE AQUI>
\end{keywords}

% Sumário
% Comente para ocultar
\tableofcontents

% Lista de figuras
% Comente para ocultar
\listoffigures

% Lista de tabelas
% Comente para ocultar
\listoftables

%%
%% Parte textual
%%
\mainmatter

% É aconselhável criar cada capítulo em um arquivo à parte, digamos
% "capitulo1.tex", "capitulo2.tex", ... "capituloN.tex" e depois
% incluí-los com:
% \include{capitulo1}
% \include{capitulo2}
% ...
% \include{capituloN}

\backmatter

% Apêndices
% Comente se não houver apêndices
\appendix

% É aconselhável criar cada apêndice em um arquivo à parte, digamos
% "apendice1.tex", "apendice.tex", ... "apendiceM.tex" e depois
% incluí-los com:
% \include{apendice1}
% \include{apendice2}
% ...
% \include{apendiceM}


% Bibliografia
% É aconselhável utilizar o BibTeX a partir de um arquivo, digamos "biblio.bib".
% Para ajuda na criação do arquivo .bib e utilização do BibTeX, recorra ao
% BibTeXpress em www.cin.ufpe.br/~paguso/bibtexpress
\bibliographystyle{abnt-alf}
\bibliography{biblio}

%% Fim do documento
\end{document}
